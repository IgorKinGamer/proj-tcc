% Seção de conceitos
\section{Estruturas de dados}

%Esta seção lista alguns conceitos tradicionais e outros que serão usados especificamente neste trabalho.
%Esta seção apresenta alguns conceitos que serão utilizados neste trabalho.

%Árvores (estrutura de dados) - representar hierarquias de memória
Uma \textit{árvore} é uma estrutura de dados que define uma hierarquia, logo, é conveniente para representar hierarquias de memória.

%Nós e ligações
Árvores são compostas por nós e ligações que relacionam esses nós.
Cada nó possui dados que dependem do que a árvore representa e de com que propósito ela será usada.
Cada ligação relaciona dois nós, um com papel de \textit{pai}, e o outro com papel de \textit{filho}.
Ou seja, uma ligação estabelece que um nó $a$ é pai de outro nó $b$, e, equivalentemente, que o nó $b$ é filho do nó $a$.

Em uma árvore, existe um único nó que não possui pai, o qual é chamado de \textit{nó raiz} (ou simplesmente \textit{raiz}).
Todos os demais nós possuem exatamente um pai.
%Cada nó conectado 
%Pode ter qualquer quantidade de filhos; se nenhum, folha.
Cada nó pode ter qualquer quantidade de filhos.
Se não possui nenhum, é chamado de \textit{nó folha} (ou \textit{folha}). %, caso contrário, de \textit{nó interno}.
%Filhos ordenados
Os filhos de um nó são ordenados.
Essa ordem não necessariamente reflete algum atributo ou característica daquilo que os nós estão representando.

%Para se realizar operações sobre essa estrutura de árvore
Para se realizar operações sobre essa estrutura de árvore, é possível seguir as ligações para descobrir os nós adjacentes (pai e filhos).
Visto que árvores geralmente são visualizadas com o nó raiz no topo, \textit{subir} uma ligação significa obter o pai de algum nó, e \textit{descer}, obter um dos filhos de um nó.
%O pai de um nó é obtido \textit{subindo} a ligação entre ele e seu pai
%Seguir(?) uma ligação

%Ancestral e descendente
%Ancestral - Todos os nós aos quais se pode chegar subindo ligações
Todos os nós que podem ser obtidos subindo, em sequência, uma ou mais ligações a partir de um nó são chamados de \textit{ancestrais} desse nó.
Semelhantemente, nós que podem ser obtidos apenas descendo ligações são \textit{descendentes}.
%Uma árvore não possui ciclos -- 
Nenhum nó é ancestral de si mesmo.

%Níveis (nível/profundidade de um nó)
Uma árvore é dividida em \textit{níveis}, cada um composto por um ou mais nós.
O nível 0 é composto pela raiz.
O nível $i$ é composto por todos os nós que podem ser obtidos descendo $i$ ligações em sequência a partir da raiz.
O nível em que um nó está é também chamado de sua \textit{profundidade}.
%Ordem?
%A ordem dos filhos de cada nó da árvore define implicitamente uma ordem em cada nível.

%O grau de um nó é a quantidade de filhos que ele possui
A quantidade de filhos que um nó possui é chamada de \textit{grau do nó}.
%- Grau de um nível: o maior entre os graus dos nós de um nível.
O maior entre os graus dos nós que compõe um nível é chamado de \textit{grau do nível}.
%É comum os graus de todos os nós de um nível serem iguais, mas, em hierarquias assimétricas(?), isso não ocorre.


%--- Encaixar ---

%Assimetria - hwloc
%Simetria - todas folhas no mesmo nível (e com graus diferentes?)

%Nós de um mesmo nível  graus diferentes
