\chapter{Implementação}

Esta seção apresenta como o projeto hwloc foi analisado, as considerações feitas com relação ao desempenho e as implementações resultantes.
Todo o código foi desenvolvido na linguagem C++.

O hwloc possui diversas funções de percorrimento.
Estas permitem acessar nós da árvore que representa a hierarquia de forma absoluta ou relativa a outros nós.
Por exemplo, é possível encontrar o nó com um determinado índice dentro de um dado nível,
ou, a partir de algum nó, o próximo no mesmo nível.
\tratar{Listar funções}

Essas funções foram analisadas quanto à complexidade com o objetivo de identificar pontos que poderiam ser melhorados do ponto de vista do desempenho.
Essa análise revelou que, em geral, elas têm tempo constante ($\bigO(1)$) ou linear na altura da árvore (\Oalt), que é o mesmo que $\bigO(\log N)$, onde $N$ é a quantidade de nós da árvore.
Além disso, foi analisado um projeto de código aberto que utiliza o hwloc, HieSchella \cite{HieSchella},
cujo objetivo é prover portabilidade de desempenho, característica presente quando se consegue que
uma mesma aplicação rode em diferentes plataformas utilizando os núcleos com eficiência.
Foram identificadas as chamadas mais importantes a funções do hwloc no HieSchella para se ter uma referência
de quais funções são mais relevantes para o desempenho dentro de um projeto real.

Diante dessas considerações, a função que encontra o ancestral comum mais próximo (\fACMP) entre dois nós foi escolhida como alvo de otimizações.
Ela é uma das funções implementadas no hwloc com complexidade \Oalt\ (ou $\bigO(\log N)$) e está entre as de uso mais significativo no HieSchella.
Na seção a seguir, será discutido como essa função poderia ser implementada de forma mais eficiente e quais as implicações de diferentes abordagens.


\section{Implementações da função \fACMP}

A função que encontra o ancestral comum mais próximo recebe dois nós como entrada (e possivelmente algumas estruturas adicionais, se houver necessidade)
e retorna um nó (o ancestral) como saída.
Cada par de nós em uma árvore tem exatamente um ancestral comum mais próximo.
Existem algumas situações especiais.
Em geral, o caso em que os dois nós são o mesmo (\fACMP($a$, $a$)) não será tratado.
Se um nó $a$ for ancestral de outro nó $b$, \fACMP($a$, $b$) terá como resultado $a$.

\subsection{Abordagem inicial}

A maneira provavelmente mais intuitiva de se descobrir o ACMP é ``subir'' pela árvore, isto é, a partir dos dois nós dos quais se deseja encontrar o ACMP,
seguir as ligações em direção aos pais até se chegar ao mesmo nó.
Por ser um método que utiliza apenas a estrutura de árvore em si, sem adição de outras informações, será referido como método simples.
Sua complexidade é $\bigO(\log N)$.
Para que esse método funcione, é necessário subir de forma sincronizada -- a cada passo, devem-se comparar ancestrais dos dois nós iniciais que estejam no mesmo nível.
No caso do hwloc, o algoritmo se torna um pouco mais complicado, pois ele trata hierarquias assimétricas (pode haver ramos sem nó em algum nível).
Neste caso, mesmo que se tenham dois nós no mesmo nível, seus pais podem estar em níveis diferentes.
Um ponto que pode afetar o desempenho deste algoritmo é o fato de que é necessário acessar cada nó
(os nós iniciais e todos os seus ancestrais até o ACMP), e os nós estão espalhados pela memória.

\tratar{(Algoritmo ACMP do hwloc)}

Considerando a estrutura de árvore apenas, a única informação que relaciona um nó aos seus ancestrais são as ligações.
% entre um nó e seus filhos (ou, no outro sentido, entre cada nó e seu pai).
%Assim, todas as sequências de uma ou mais ligações de filho para pai a partir de um nó (ou seja,
%entre o nó e seu pai, entre este e o pai dele, e assim por diante) definem os ancestrais do nó.
Isso indica que outras estruturas associadas aos nós ou à árvore como um todo se fazem necessárias
para ser possível encontrar o ACMP com algum método além do simples.
Podemos considerar a seguinte ideia para encontrar outra maneira de implementar a função de ancestral comum:
Para uma dada árvore que representa uma topologia,
\begin{enumerate}
	\item Atribuir um valor (chamado de ID) a cada nó da árvore;
	\item Definir uma função \ACMPIDs\ que receba o ID de dois nós distintos e tenha como resultado o nó ACMP.
\end{enumerate}
Esses IDs (em conjunto com outras informações associadas a cada nó individualmente ou à árvore como um todo conforme necessário)
podem estabelecer alguma relação entre um nó e seus ancestrais além da que já existe por meio das ligações da árvore.

Idealmente, essa função \ACMPIDs\ deveria ser \textit{processada} em tempo constante.
Ou seja, é necessário encontrar um algoritmo que seja executado em tempo constante nos processadores modernos.
No entanto, é preciso lembrar que, mesmo que a quantidade de instruções executadas pelo processador seja constante,
a maneira como a memória é acessada pode aumentar o tempo de execução, especialmente quando há outras tarefas fazendo uso da memória, o que deve acontecer em cenários reais.
\tratar{(? Isso pode ser visto... (exemplo à frente com matriz?))}

Com isso em mente, podemos definir tal função usando as operações básicas encontradas no conjunto de instruções das arquiteturas atuais,
tais como as operações aritméticas e operações lógicas bit-a-bit.

\subsection{Matriz}

Uma possibilidade é relacionar cada par de nós ao seu ACMP por meio de uma matriz
em que cada linha representa um nó da árvore, assim como cada coluna, e o cruzamento contém o ACMP entre o nó da linha e o nó da coluna.
Para isso, pode-se atribuir a cada um dos $N$ nós da árvore um ID único entre $0$ e $N-1$ e usar esses IDs como índices na matriz,
que terá, na posição $A(i,j)$, o ACMP entre o nó de ID $i$ e o nó de ID $j$.
No entanto, esta é uma estratégia ingênua, pois esse espaço $\bigO(N^2)$ ocupado na memória resultaria em problemas como sujar a cache da aplicação.
A Tabela~\ref{tab:dif_arvore_matriz} sumariza a diferença entre a utilização de uma árvore (método simples) e de uma matriz (método matricial).
Visto que a matriz é simétrica, apenas cerca de metade dela precisa realmente ser armazenada.
Esta otimização foi utilizada na implementação dos testes de desempenho, conforme o Algoritmo~\ref{alg:matriz}.
Isso, no entanto, não altera a complexidade espacial.

% Tabela comparando árvore com matriz
\begin{table}[htb]
	\centering
	\caption{Complexidade das representações com árvore e com matriz}
	\label{tab:dif_arvore_matriz}
	\begin{tabular}{ccc}
		\toprule
		& Árvore & Matriz \\
		\midrule
		\begin{tabular}{@{}c@{}}Acesso\\(encontrar ACMP)\end{tabular}
			& $\bigO(\log N)$
			& $\bigO(1)$ \\
		\begin{tabular}{@{}c@{}}Espaço ocupado\\na memória\end{tabular}
			& $\bigO(N)$
			& $\bigO(N^2)$ \\
		\bottomrule
	\end{tabular}
\end{table}

% Matriz
\begin{algorithm}[h]
	\caption{\textsc{MatrizACMP} -- \fACMP\ usando uma matriz}
	\label{alg:matriz}
	
	\newcommand \linha {\mathrm{linha}}
	\newcommand \coluna {\mathrm{coluna}}
	
	\Entrada{Dois nós $a$ e $b$ \newline
	         Uma matriz $M$ que possui na linha $i$ e coluna $j$ ($M(i, j)$),
	          $i > j$, o ACMP entre os nós de ID $i$ e $j$}
	\Saida{O ACMP entre $a$ e $b$}
	
	%\BlankLine
	%\tcp{A matriz é simétrica, elementos acima da diagonal principal não são utilizados}
	
	\BlankLine
	%Matriz$\langle$Nó$\rangle$ $M$, Nó \no{a}, Nó \no{b}
	$\linha  \gets \max(\id{a}, \id{b})$ \\
	$\coluna \gets \min(\id{a}, \id{b})$ \\
	\Retorna $M(\linha, \coluna)$ \\
	%\EndFunction
	
\end{algorithm}


\subsection{Função de espalhamento}

Outra possibilidade foi idealizada, dividindo a função \ACMPIDs\ em dois passos:
\begin{enumerate}
	\item dados os IDs de dois nós, descobrir o ID do ancestral;
	\item encontrar o nó que possui esse ID.
\end{enumerate}
Em linhas gerais, o funcionamento do primeiro passo se baseia no seguinte:
%- O comprimento de um ID é a quantidade de bits da sua representação em binário, descartando zeros à esquerda.
O ID de um nó é formado por um ou mais bits seguidos do ID do seu pai, de modo que, dados dois nós \no{a} e \no{b}, \no{b} descendente de \no{a},
os bits menos significativos de \no{b} são iguais ao ID de \no{a}.
Desse modo, dados dois descendentes de um nó \no{c}, todos os bits menos significativos deles que coincidem
(todos os que vêm antes do primeiro que difere) são iguais ao ID de \no{c} (ignorando zeros à esquerda).
Usando apenas as operações que podem ser vistas nas linhas 1 -- 4 do Algoritmo~\ref{alg:novo} mais adiante,
para as quais existem instruções que tomam poucos ciclos nas arquiteturas atuais, pode-se descobrir o ID do ACMP.
A quantidade de instruções é fixa, portanto, a complexidade é constante.

Para descrever como os IDs são formados, as seguintes definições são necessárias:
\begin{itemize}
	\item \id{a} é o ID do nó $a$.
	\item \idstr{a} é uma cadeia (\textit{string}) de bits correspondente ao \id{a} em binário (com o bit menos significativo na última posição). O tamanho depende do nível de $a$, como será especificado adiante.
\end{itemize}
Os IDs, então, são definidos da seguinte forma:
\begin{itemize}
	\item A raiz tem id 0, e \idstr{\mathit{raiz}} é a cadeia vazia.
	\item Quanto aos demais, para cada nó $a$,
		$$\idstr{a} = x \parallel \idstr{pai(a)}$$
	onde $\parallel$ é a concatenação e $x$ é uma cadeia cujo tamanho é o grau do nível de $\pai(a)$.
	Se a é o i-ésimo filho de seu pai, $x$ possui 1 na i-ésima posição da direta para a esquerda e 0 nas demais.
\end{itemize}

Assim, as cadeias correspondentes aos IDs de todos os nós de um nível têm o mesmo tamanho, que é o somatório dos graus dos níveis anteriores.
A Figura~\ref{img:exemplo_ids} apresenta os IDs atribuídos aos nós de uma árvore.
O \tratar{Algoritmo X} mostra como o ID do ACMP é obtido a partir do ID de dois nós.

\begin{figure}
	\centering
	\caption{Exemplo de árvore com IDs atribuídos aos nós}
	\label{img:exemplo_ids}
	\begin{tikzpicture}[no/.style={draw, circle, minimum size=1cm, align=center}]
	%[estado/.style={circle,minimum size=1cm}]
	%\node [estado] (IUA)    at ({360/\n * (3+.5)}:\raio) {I U A};
		% Árvore
		\node [no] (a) {$a$}
			[level distance=2cm,
			level 1/.style={sibling distance=30mm},
			level 2/.style={sibling distance=15mm}]
			child {node [no] (b) {b}
				child {node [no] (e) {e}}
				child {node [no] (f) {f}} }
			child {node [no] (c) {c}
				child {node [no] (g) {g}}
				child {node [no] (h) {h}} }
			child {node [no] (d) {d}
				child {node [no] (i) {i}}
				child {node [no] (j) {j}} }
		;
		% IDs
		\newcommand \caa[1] {\textcolor{red!50!black}{#1}}
		\newcommand \cab[1] {\textcolor{red!50!red}{#1}}
		\newcommand \cac[1] {\textcolor{red!50!white}{#1}}
		\newcommand \cba[1] {\textcolor{blue!75!black}{#1}}
		\newcommand \cbb[1] {\textcolor{blue!75!white}{#1}}
		\DeclareDocumentCommand \labelId {m m m O{right} O{0}}
			{\node [align=center, node distance=#5, #4=of #1]
			{$\id{#1} = #2$\\$\idstr{#1} = #3$}}
		% Folha
		\DeclareDocumentCommand \labelIdF {m m m O{0}}
			{\labelId{#1}{#2}{#3}[below][#4]}
		% Folha mais baixa
		\DeclareDocumentCommand \labelIdFB {m m m}
			{\labelIdF{#1}{#2}{#3}[1]}
		\labelId{a}{0}{{\color{gray} (vazio)}};
			\labelId{b}{1}{\caa{001}};
				\labelIdF{e}{9}{\cba{01}\ \caa{001}};
				\labelIdFB{f}{17}{\cbb{10}\ \caa{001}};
			\labelId{c}{2}{\cab{010}};
				\labelIdF{g}{10}{\cba{01}\ \cab{010}};
				\labelIdFB{h}{18}{\cbb{10}\ \cab{010}};
			\labelId{d}{4}{\cac{100}};
				\labelIdF{i}{12}{\cba{01}\ \cac{100}};
				\labelIdFB{j}{20}{\cbb{10}\ \cac{100}};
	\end{tikzpicture}
\end{figure}
 % Explicação de como os IDs são formados

\begin{algorithm}
	\caption{\Novo\ -- \fACMP\ usando as novas estruturas}
	\label{alg:novo}
	
	\newcommand \imgN {E}
	
	\Entrada{Dois nós $a$ e $b$ \newline
	         %Opa \newline
	         Uma função \bpn, que retorna o nível
	          do ancestral comum entre dois nós
	          dado um número com apenas um bit 1,
	          na primeira posição em que o ID dos nós difere \newline
	         A função \textsc{Espalhamento} (Algoritmo \ref{alg:espalha})
	          com dados da árvore em questão \newline
	         Uma matriz (vetor de vetores) $\imgN$,
	          onde o vetor $\imgN(i)$ possui os nós do nível $i$
	          distribuídos pela função \textsc{Espalhamento}($i$, id)}
	\Saida{O ACMP entre $a$ e $b$}
	\BlankLine
	
	%Nota: Processadores atuais possuem uma instrução cujo resultado
	%      é a posição do primeiro bit 1 em um número,
	%      o qual pode ser usado para acessar um arranjo
	%\BlankLine
	
	\LinesNumbered
	%\Function{ancestral}{Estruturas $\var{E}$, No $a$, No $b$}
	%ID dif, bit, \var{masc}, id;
	\tcp{Bits que diferem}
	$\var{dif} \gets \id{a} \OuExBit \id{b}$ \\
	\tcp{Bit 1 apenas na primeira posição em que os IDs diferem}
	$\var{bit} \gets \var{dif} \EBit (-\var{dif})$ \\
	\tcp{Todos os bits antes do primeiro diferente}
	$\var{\var{masc}} \gets \var{bit} - 1$ \\
	%//\var{masc} = (((\var{dif}-1) ^ \var{dif}) >> 1);
	\tcp{Id do ancestral comum mais próximo}
	$\var{id} \gets \id{a} \EBit \var{masc}$ \\
	
	\BlankLine
	\tcp{Encontra o nó}
	$\var{nível} \gets \bpn(\var{bit})$ \\
	$\var{pos}   \gets \textsc{Espalhamento}(\var{nível}, \var{id})$ \\
	\Retorna $\imgN(\var{nível}, \var{pos})$ \\
	
	%\tcp{Arranjo com os nós já distribuídos pela função de espalhamento}
	%$\var{imagemNivel} \gets \mathrm{E.imagemNiveis}(nivel)$ \\
	%$\var{imagemNó} \gets
	%	\mathrm{E.dadosFuncao}[\var{nivel}].aplicar(\var{id})$ \\
	%\Retorna $\var{imagemNivel}[\var{imagemNó}]$ \\
\end{algorithm}

% Espalhamento
\begin{algorithm}
	\caption{\Espalha\ -- Função de espalhamento}
	\label{alg:espalha}
	
	\Entrada{Um nível (inteiro não negativo) $n$ \newline
	         O id de um nó}
	\Saida{A posição em que o nó deve ficar no vetor do seu nível}
	\BlankLine
	
	Nota: Os valores $\var{ad}_{n}$, $\var{m}_{n}$,
	                 $\var{mult}_{n}$ e $\var{desl}_{n}$
	      dependem da árvore com a qual se usará esta função
	      e do nível recebido ($n$).
	      $x \Desl b$ é o valor $x$ deslocado $b$ bits para a direita,
	      equivalente a $\lfloor x/(2^b)\rfloor$. \\
	\BlankLine
	
	\tcp{Minimiza o resultado ${}\bmod m_{n}$ para minimizar o tamanho do vetor}
	$\var{v} \gets \var{id} + \var{ad}_{n}$ \\
	\tcp{Equivalente a $\var{v} \bmod m_{n}$}
	\Retorna $\var{v} - ((\var{v} * \var{mult}_{n})
		\Desl \var{desl}_{n}) * m_{n}$ \\
\end{algorithm}


Falta, então, apenas o segundo passo, o de encontrar o nó a partir do ID.
Algumas opções para isso seriam:
\begin{itemize}
	\item Usar os IDs como índices em um arranjo: Seria simples, mas impraticável -- poderiam ser necessários arranjos com milhões de posições (devido a como os IDs são formados) e apenas algumas centenas ocupadas.
	\item Usar uma função de espalhamento (\textit{hash}): a maneira mais simples seria apenas aplicar a operação módulo com algum $m$ ($\mathrm{id} \bmod m$).
	No entanto, podem haver colisões (dois IDs diferentes podem ser congruentes módulo $m$).
	Isso pode ser tratado, mas acarretaria acessos adicionais à memória, o que não é desejável.
	\tratar{Exemplo}
	\item Buscar uma função que não cause colisões: o mesmo que a função de espalhamento, porém utilizando um valor de módulo que não cause colisões.
	Pode exigir arranjos cujo tamanho é algumas vezes maior que a quantidade de nós, mas aparenta compensar quando comparado a tratar colisões de outros modos.
	\tratar{Exemplo}
	Este foi o método escolhido.
\end{itemize}

As linhas 5~--~7 do Algoritmo~\ref{alg:novo} apresentam como este método funciona.
Quanto à implementação da função \bpn, os processadores atuais possuem uma instrução que encontra a posição do primeiro bit 1 em um número, a qual pode ser usada como índice em um pequeno arranjo contendo o devido resultado da função \bpn.
Se os IDs têm até $b$ bits, esse arranjo precisar ter $b$ posições.
A função \Espalha\ (Algoritmo~\ref{alg:espalha}) utiliza dados específicos para a árvore, que devem ser descobertos previamente.

Testes feitos indicaram que a operação mais custosa no \Novo\ era o módulo, usado na função \Espalha.
No entanto, existem técnicas para realizar de maneira mais barata a divisão com denominador previamente conhecido \cite{reciproco}, e com o resultado da divisão pode-se calcular o módulo.
Esta otimizações são usadas, por exemplo, por compiladores.
Aqui, no entanto, a ideia não é realizá-las em tempo de compilação, mas quando se está montando as estruturas para uma hierarquia específica.
Isso continua sendo vantajoso pois a montagem ocorre apenas uma vez e esta operação de módulo com o mesmo valor será realizada uma grande quantidade de vezes.
Portanto, é possível substituir a operação de módulo por outras operações que se mostraram mais baratas, a saber, duas multiplicações, um deslocamento e uma subtração.
%Com esta alteração, os tempos diminuiram consideravelmente, conforme mostra a tabela [X])

Para se encontrarem valores apropriados para a função \Espalha, são testados valores cada vez maiores para $m$, até que não haja colisões.
Ao se encontrar um $m$ válido, se busca um $\mathit{ad}$ que minimize o maior resultado de $\mathrm{id} \bmod m$, minimizando o tamanho necessário do arranjo.
Então, são descobertos os valores de $\mathit{mult}$ e $\mathit{desl}$, usando a técnica descrita em \citeonline{reciproco}, para evitar a operação de módulo.

%\tratar{Estruturas foram testadas}
A corretude das estruturas utilizadas na implementação do \Novo\ foi testada.
O programa de testes cria uma árvore e compara o resultado do \Simples\ com o do \Novo\ para algumas buscas, além de verificar se todos os nós estão realmente na posição retornada pela função \Espalha.
Também é causada uma mutação em uma árvore e é verificado se o erro é detectado por esse código de verificação.


%Detalhes de como os IDs são formados
\tratar{Pode falhar se os dois nós de entrada são, na verdade, o mesmo nó. If resolve
Funciona em casos de hierarquias assimétricas, simples não
Limitação: Quantidade de bits necessária - Solução parcial: 'Esconder' níveis cujos nós têm só um filho (níveis de grau um).}



%SEÇÃO Futuro

%Melhorias e alternativas não testadas:
%- Eliminar multiplicações (mas testes feitos mostram que não têm grande custo)
%- Todos os nós em um só arranjo
%- Máscaras Ou e OuExclusivo; Outras maneiras de diminuir o tamanho

%HieSchella - hwloc\_get\_ancestor\_obj\_by\_type
