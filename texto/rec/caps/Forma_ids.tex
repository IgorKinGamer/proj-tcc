Para descrever como os IDs são formados, as seguintes definições são necessárias:
\begin{itemize}
	\item \id{a} é o ID do nó $a$.
	\item \idstr{a} é uma cadeia (\textit{string}) de bits correspondente ao \id{a} em binário (com o bit menos significativo na última posição). O tamanho depende do nível de $a$, como será especificado adiante.
\end{itemize}
Os IDs, então, são definidos da seguinte forma:
\begin{itemize}
	\item A raiz tem id 0, e \idstr{\mathit{raiz}} é a cadeia vazia.
	\item Quanto aos demais, para cada nó $a$,
		$$\idstr{a} = x \parallel \idstr{pai(a)}$$
	onde $\parallel$ é a concatenação e $x$ é uma cadeia cujo tamanho é o grau do nível de $\pai(a)$.
	Se a é o i-ésimo filho de seu pai, $x$ possui 1 na i-ésima posição da direta para a esquerda e 0 nas demais.
\end{itemize}

Assim, as cadeias correspondentes aos IDs de todos os nós de um nível têm o mesmo tamanho, que é o somatório dos graus dos níveis anteriores.
A Figura~\ref{img:exemplo_ids} apresenta os IDs atribuídos aos nós de uma árvore.
O \tratar{Algoritmo X} mostra como o ID do ACMP é obtido a partir do ID de dois nós.

\begin{figure}
	\centering
	\caption{Exemplo de árvore com IDs atribuídos aos nós}
	\label{img:exemplo_ids}
	\begin{tikzpicture}[no/.style={draw, circle, minimum size=1cm, align=center}]
	%[estado/.style={circle,minimum size=1cm}]
	%\node [estado] (IUA)    at ({360/\n * (3+.5)}:\raio) {I U A};
		% Árvore
		\node [no] (a) {$a$}
			[level distance=2cm,
			level 1/.style={sibling distance=30mm},
			level 2/.style={sibling distance=15mm}]
			child {node [no] (b) {b}
				child {node [no] (e) {e}}
				child {node [no] (f) {f}} }
			child {node [no] (c) {c}
				child {node [no] (g) {g}}
				child {node [no] (h) {h}} }
			child {node [no] (d) {d}
				child {node [no] (i) {i}}
				child {node [no] (j) {j}} }
		;
		% IDs
		\newcommand \caa[1] {\textcolor{red!50!black}{#1}}
		\newcommand \cab[1] {\textcolor{red!50!red}{#1}}
		\newcommand \cac[1] {\textcolor{red!50!white}{#1}}
		\newcommand \cba[1] {\textcolor{blue!75!black}{#1}}
		\newcommand \cbb[1] {\textcolor{blue!75!white}{#1}}
		\DeclareDocumentCommand \labelId {m m m O{right} O{0}}
			{\node [align=center, node distance=#5, #4=of #1]
			{$\id{#1} = #2$\\$\idstr{#1} = #3$}}
		% Folha
		\DeclareDocumentCommand \labelIdF {m m m O{0}}
			{\labelId{#1}{#2}{#3}[below][#4]}
		% Folha mais baixa
		\DeclareDocumentCommand \labelIdFB {m m m}
			{\labelIdF{#1}{#2}{#3}[1]}
		\labelId{a}{0}{{\color{gray} (vazio)}};
			\labelId{b}{1}{\caa{001}};
				\labelIdF{e}{9}{\cba{01}\ \caa{001}};
				\labelIdFB{f}{17}{\cbb{10}\ \caa{001}};
			\labelId{c}{2}{\cab{010}};
				\labelIdF{g}{10}{\cba{01}\ \cab{010}};
				\labelIdFB{h}{18}{\cbb{10}\ \cab{010}};
			\labelId{d}{4}{\cac{100}};
				\labelIdF{i}{12}{\cba{01}\ \cac{100}};
				\labelIdFB{j}{20}{\cbb{10}\ \cac{100}};
	\end{tikzpicture}
\end{figure}
