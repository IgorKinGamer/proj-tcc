\chapter{Fundamentação Teórica}
\label{cap:fundamentacao_teorica}
\acresetall


%Uma parte muito importante das hierarquias a ser considerada em questões de desempenho são as memórias cache.
As hierarquias de memória possuem várias características que afetam o desempenho de aplicações e precisam, portanto, ser conhecidas ao se trabalhar com aplicações de alto desempenho.
Nas seções a seguir são apresentados alguns conceitos importantes nesse contexto, e, ao final, alguns conceitos relacionados a estruturas de dados relevantes.



\section{Memórias Cache}
\label{sec:memorias_cache}

Caches são memórias com o próposito de diminuir o tempo médio de acesso aos dados. Mais especificamente, é comum haver dois ou três níveis de cache entre as unidades de processamento e a memória principal. Elas são construídas com tecnologias que as tornam mais rápidas que outros níveis abaixo \cite{Patterson}. Porém, essa velocidade vem em troca de custo mais elevado. Por isso, elas têm espaço de armazenamento menor, além de que memórias maiores podem ter sua velocidade de acesso diminuída, o que também motiva a existência de vários níveis de cache.

O princípio de sua funcionalidade é manter à disposição dos programas, de forma rápida, aqueles dados dos quais eles precisam ou que estão usando no momento. Esses dados são disponibilizados conforme a capacidade de armazenamento da cache. Esta comumente é menor que o conjunto de todos os dados sobre os quais o programa opera, resultando na necessidade de remover alguns dados para acomodar outros.

\subsection{Localidade Espacial}
\label{subsec:localidade_espacial}

Quando um programa referencia determinada posição da memória, é comum que logo em seguida os dados nas posições de memória adjacentes sejam necessários também. Assim é definida a localidade espacial: endereços próximos tendem a ser referenciados um após o outro com pouco tempo de diferença \cite{Patterson}. As caches levam isso em conta para beneficiar as aplicações, trazendo dos níveis de memória abaixo não só o dado requisitado, mas também os dados que o rodeiam, constituindo um bloco. Assim, do ponto de vista da cache, a memória é uma sequência de blocos que, quando necessários, são carregados em algum espaço disponível, ou substituem um bloco carregado previamente se não houver espaços disponíveis.

Analisar as caches ajuda a entender por que um esquema que usasse estruturas muito espaçosas para reduzir a quantidade de operações e acessos poderia não funcionar bem.
No pior caso, acessos consecutivos poderiam ser todos a blocos diferentes, ocasionando o custo de trazer cada um para a cache e resultando na poluição da cache das aplicações, ou seja, diversos blocos com dados das aplicações seriam substituídos, tornando maior o tempo para acessá-los na próxima vez.


\section{\textit{Non-Uniform Memory Access}}
\label{sec:numa}

\textit{Non-Uniform Memory Access} (NUMA) é um tipo de multiprocessador com espaço de endereçamento único em que diferentes partes da memória estão mais próximas de alguns núcleos do que dos outros \cite{Patterson}.
Deste modo, o tempo de acesso depende do núcleo e da memória acessada.
Sistemas desse tipo possuem boa escalabilidade, pois é possível adicionar nós sem prejudicar o tempo de acesso dos núcleos às memórias mais próximas.
Estas características diferem das de outro tipo de multiprocessador existente, \textit{Uniform Memory Access} (UMA), em que o tempo de acesso independe do núcleo e da memória, e não há a vantagem de memórias locais com tempo de acesso reduzido.
Um problema com arquiteturas UMA é a contenção: vários núcleos disputam pelo barramento para acessar a memória, podendo fazer com que alguns fiquem ociosos enquanto esperam, diminuindo o desempenho.
Além disso, a escalabilidade é reduzida -- ao aumentar a quantidade de núcleos, o tempo de acesso à memória também cresce, seja qual for a região acessada, podendo se tornar inaceitável.

O modo como os nós NUMA são interconectados determina os custos de comunicação entre quaisquer dois núcleos em nós distintos.
Além disso, ele define os níveis que existirão a mais na hierarquia, ou seja, a sua profundidade.
Essas informações são de grande valia e devem ser fornecidas por representações da topologia com precisão.
Mesmo poucos níveis a mais na rede de interconexão podem resultar em diferenças de desempenho significativas \cite{MPITopFunc}.


% Seção de conceitos
\section{Estruturas de dados}

%Esta seção lista alguns conceitos tradicionais e outros que serão usados especificamente neste trabalho.
%Esta seção apresenta alguns conceitos que serão utilizados neste trabalho.

%Árvores (estrutura de dados) - representar hierarquias de memória
Uma \textit{árvore} é uma estrutura de dados que define uma hierarquia, logo, é conveniente para representar hierarquias de memória.

%Nós e ligações
Árvores são compostas por nós e ligações que relacionam esses nós.
Cada nó possui dados que dependem do que a árvore representa e de com que propósito ela será usada.
Cada ligação relaciona dois nós, um com papel de \textit{pai}, e o outro com papel de \textit{filho}.
Ou seja, uma ligação estabelece que um nó $a$ é pai de outro nó $b$, e, equivalentemente, que o nó $b$ é filho do nó $a$.

Em uma árvore, existe um único nó que não possui pai, o qual é chamado de \textit{nó raiz} (ou simplesmente \textit{raiz}).
Todos os demais nós possuem exatamente um pai.
%Cada nó conectado 
%Pode ter qualquer quantidade de filhos; se nenhum, folha.
Cada nó pode ter qualquer quantidade de filhos.
Se não possui nenhum, é chamado de \textit{nó folha} (ou \textit{folha}). %, caso contrário, de \textit{nó interno}.
%Filhos ordenados
Os filhos de um nó são ordenados.
Essa ordem não necessariamente reflete algum atributo ou característica daquilo que os nós estão representando.

%Para se realizar operações sobre essa estrutura de árvore
Para se realizar operações sobre essa estrutura de árvore, é possível seguir as ligações para descobrir os nós adjacentes (pai e filhos).
Visto que árvores geralmente são visualizadas com o nó raiz no topo, \textit{subir} uma ligação significa obter o pai de algum nó, e \textit{descer}, obter um dos filhos de um nó.
%O pai de um nó é obtido \textit{subindo} a ligação entre ele e seu pai
%Seguir(?) uma ligação

%Ancestral e descendente
%Ancestral - Todos os nós aos quais se pode chegar subindo ligações
Todos os nós que podem ser obtidos subindo, em sequência, uma ou mais ligações a partir de um nó são chamados de \textit{ancestrais} desse nó.
Semelhantemente, nós que podem ser obtidos apenas descendo ligações são \textit{descendentes}.
%Uma árvore não possui ciclos -- 
Nenhum nó é ancestral de si mesmo.

%Níveis (nível/profundidade de um nó)
Uma árvore é dividida em \textit{níveis}, cada um composto por um ou mais nós.
O nível 0 é composto pela raiz.
O nível $i$ é composto por todos os nós que podem ser obtidos descendo $i$ ligações em sequência a partir da raiz.
O nível em que um nó está é também chamado de sua \textit{profundidade}.
%Ordem?
%A ordem dos filhos de cada nó da árvore define implicitamente uma ordem em cada nível.

%O grau de um nó é a quantidade de filhos que ele possui
A quantidade de filhos que um nó possui é chamada de \textit{grau do nó}.
%- Grau de um nível: o maior entre os graus dos nós de um nível.
O maior entre os graus dos nós que compõe um nível é chamado de \textit{grau do nível}.
%É comum os graus de todos os nós de um nível serem iguais, mas, em hierarquias assimétricas(?), isso não ocorre.


%--- Encaixar ---

%Assimetria - hwloc
%Simetria - todas folhas no mesmo nível (e com graus diferentes?)

%Nós de um mesmo nível  graus diferentes

