\chapter{Introdução}
\label{cap:introducao}
\acresetall


Atualmente, arquiteturas de computadores são construídas de forma hierárquica quanto à memória.
Essa hierarquia diz respeito à passagem de dados entre os diferentes níveis, ou seja, quais caminhos existem para que dados sejam comunicados entre pontos dessa hierarquia.
O que motiva sua existência é o fato de que diferentes tipos de memória possuem tamanhos, velocidades de acesso e custos distintos, e ela permite que o espaço das memórias maiores esteja disponível sem que se perca a velocidade das menores e mais rápidas \cite{Patterson}.
Níveis de memória mais baixos possuem maior capacidade de armazenamento, porém seu tempo de acesso é maior.
Parte dessa hierarquia é composta por um ou mais níveis de cache, memórias com capacidade reduzida, mas maior velocidade, permitindo que, em um dado momento, um conjunto de dados qualquer possa ser acessado mais rapidamente.
Sobre essas hierarquias e fazendo uso delas estão as unidades de processamento, acessando e operando sobre os dados em memória.
Quanto mais próximo for o nível de memória em que esses dados estiverem, menor o tempo de acesso.
Quando essas unidades precisam se comunicar entre si, elas fazem uso da hierarquia de memória.

A hierarquia pode ser organizada de várias formas, podendo se tornar complexa e de grande profundidade.
Uma possibilidade na organização é o compartilhamento de alguns níveis de cache, ou seja, duas unidades de processamento ou mais estarão acima de uma mesma cache da hierarquia.
Isso permite, por exemplo, realizar comunicações com eficiência, pois o tempo entre algum dado ser atualizado e o novo valor ser visto é determinado pelo tempo de acesso à cache compartilhada.
Uma grande variedade de organizações pode ser encontrada ao se considerar arquiteturas como multicore, em que várias unidades de processamento chamadas de núcleos (\textit{cores}) fazem parte de um mesmo circuito integrado, ou \textit{Non-Uniform Memory Access} (NUMA), onde a rede que interconecta os nós dá origem a mais níveis de hierarquia. Esta rede, por si só, pode ser organizada de várias formas distintas.
Essa organização compreendendo hierarquia de memória e unidades de processamento, na sua totalidade, define uma topologia de máquina.

A necessidade de plataformas para rodar aplicações de alto desempenho tem dado origem às diversas arquiteturas paralelas modernas existentes.
À medida que novas tecnologias surgem, se faz necessário adaptar as arquiteturas
para permitir o seu uso eficiente, de modo a se atingir alto desempenho nessas plataformas \cite{ataque}.
Suas topologias são as mais variadas, visando atender às necessidades de várias classes de aplicações com características e comportamentos distintos, que vão desde simulações científicas até o processamento de tarefas em sistemas que dão suporte a redes sociais \cite{sociais}.
Diante da crescente complexidade das topologias dessas máquinas, as suas organizações e as demais características dos elementos que compõe a hierarquia de memória são aspectos de muita relevância para o desempenho de aplicações.

Certas combinações de fatores da aplicação e da arquitetura podem resultar na melhoria ou na degradação do desempenho.
Tais fatores podem ser, por exemplo, a quantidade de dados manipulados e o tamanho das caches, que podem comportar ou não todos os dados simultaneamente, ou os padrões de troca de mensagens entre tarefas e a localização delas, além dos meios existentes para realizar essas comunicações, que podem resultar em maior ou menor eficiência \cite{Sequoia}. Um estudo de caso apresentado por \citeonline{LIKWID} mostra como uma certa distribuição de tarefas faz o desempenho cair aproximadamente pela metade.

Ainda, em arquiteturas NUMA, nas quais
%a memória é composta de várias partes que estão diretamente ligadas a nós distintos,
cada parte da memória está mais próxima de alguns nós,
%de modo que diferentes regiões da memória possuem diferentes tempos de acesso,
de modo que o tempo de acesso varia conforme a região da memória,
é importante que haja proximidade entre os dados acessados por uma thread e o núcleo em que ela reside.
Portanto, é essencial o conhecimento da topologia da máquina, que possibilita o devido ajuste das aplicações a ela, de modo a aproveitar ao máximo os recursos disponíveis.

Disso vem a necessidade de haver alguma representação da topologia para fornecer as informações necessárias sobre ela, seja diretamente às aplicações ou a outras partes do sistema, que usarão tais informações para realizar otimizações estática ou dinamicamente.
Como exemplo de uso estático, pode-se citar compilação de algoritmos com conhecimento da hierarquia \cite{Sequoia}, ou posicionamento de processos MPI \cite{hwloc2010}; e, quanto ao uso dinâmico, posicionamento de threads e dados OpenMP \cite{FGOMP}.

%[hwloc2010] http://www.open-mpi.de/papers/pdp-2010/hwloc-pdp-2010.pdf
%[FGOMP] https://hal.inria.fr/inria-00496295/document

No entanto, a disponibilização de tais informações gera custos adicionais, além de ter outras implicações relacionadas ao tamanho das estruturas de dados que podem afetar o desempenho.
Assim, é necessário que haja um compromisso entre o tempo de acesso e o espaço ocupado pela representação utilizada.
Tempos de acesso muito grandes podem acabar anulando os ganhos das otimizações.
Já se as estruturas de dados forem muito espaçosas, pode ser que não possuam boa localidade espacial, dependendo dos padrões de referência aos dados em acessos consecutivos.
Isso pode resultar em perda de desempenho ocasionada por faltas de cache, tanto no acesso às informações da topologia quanto no acesso pelas aplicações aos seus próprios dados.
Entretanto, é possível que a adição de algumas informações facilitem certas consultas sobre a topologia sem causar tais prejuízos, que é o desejado.



\section{Motivação}
\label{sec:motivacao}

Os exemplos de usos estáticos e dinâmicos dados acima, além de vários outros existentes, com o uso de benchmarks, servem como justificativa para a realização de esforços para desenvolver reprentações com as características citadas, isto é, bom tempo de acesso e uso eficiente da memória.

Para as aplicações, o ideal é que os dados estejam sempre nos níveis de cache os mais próximos possíveis, de modo que seu uso nas computações seja mais eficiente.
Diante disso, compilação com conhecimento da hierarquia \cite{Sequoia} se vale do fato de que frequentemente problemas podem ser dividos em problemas menores de tamanho variável, e ajustar esses tamanhos à capacidade das caches torna o uso delas mais efetivo, pois todos os dados usados nessas partes menores da computação caberão nelas.
Ainda, quando é possível haver vários níveis de subdivisão do problema, formando também uma espécie de hierarquia de subdivisões, os tamanhos das partes em diferentes níveis podem ser ajustados aos níveis de cache consecutivos.
Isso pode ser visualizado com facilidade no exemplo de multiplicação de grandes matrizes presente no artigo referenciado.

A velocidade de níveis de cache mais próximos também beneficia a comunicação.
Isso pode ser visto no uso de \textit{Message Passing Interface} (MPI), um padrão utilizado no desenvolvimento de programas paralelos que seguem o modelo de passagem de mensagens.
Em conjunto com dados sobre os padrões de comunicação entre processos, as informações sobre compartilhamento de caches podem ser usadas para definir um posicionamento de processos MPI que favoreça as comunicações \cite{hwloc2010}.
Outra otimização possível é o uso de metódos específicos do \textit{hardware} para realizar comunicações dentro de um nó.

No contexto de arquiteturas NUMA, para diminuir o número de acessos a memórias remotas, há a possibilidade de mover os dados para outro nó ou as threads para outros núcleos.
Uma combinação dessas opções foi desenvolvida no ForestGOMP \cite{FGOMP}, uma extensão de uma implementação de OpenMP, padrão utilizado no desenvolvimento de programas paralelos para sistemas com memória compartilhada.
Seguindo o princípio de realizar essa combinação com base nos níveis da topologia, o posicionamento dinâmico de threads e dados desenvolvido no ForestGOMP se mostrou efetivo.
Um cenário apresentado é a existência de vários conjuntos de threads e dados com grande afinidade, em que a migração de uma thread para outro núcleo só ocorreria se houvesse um nível de cache compartilhado, de modo a manter a thread próxima dos seus dados, enquanto em outros casos poderia haver a migração de todas as threads e dados relacionados.

Outro projeto que se vale do conhecimento da topologia é o HieSchella \cite{HieSchella},
cujo objetivo é prover portabilidade de desempenho, característica presente quando se consegue que
uma mesma aplicação rode em diferentes plataformas utilizando os núcleos com eficiência.
Informações sobre os custos de comunicação da plataforma são obtidos e disponibilizados para algoritmos de mapeamento de tarefas.
Alguns algoritmos de balanceamento de carga desenvolvidos utilizando o modelo de topologia disponibilizado pelo HieSchella demonstraram desempenho superior ao de outros balanceadores de carga existentes \cite{tese}.

Esses exemplos ilustram como informações sobre a hierarquia podem efetivamente ser usadas para melhorar o desempenho de aplicações que seguem modelos ou estratégias em uso real, ou seja, os benchmarks utilizados possuem características encontradas na solução de problemas reais.
Isso diz respeito a, por exemplo, padrões de comunicação ou distribuição de carga, que podem apresentar irregularidades e outras características presentes em aplicações científicas de diversas áreas.



\section{Objetivos}
\label{sec:objetivos}

Este trabalho tem como objetivo o desenvolvimento de uma representação de topologias de máquina, compreendendo as estruturas de dados utilizadas e os métodos de acesso, que mantenha o compromisso necessário entre tempo de acesso e espaço ocupado na memória pelas estruturas de dados.

\subsection{Objetivos Específicos}
\label{subsec:objetivos_especificos}

Os objetivos específicos são:
\begin{itemize}
	\item Analisar fatores relevantes para a eficiência na representação de topologias
	\item Desenvolver representações (estruturas de dados e métodos de acesso)
	\item Testar as representações desenvolvidas, por meio de experimentos em diferentes máquinas, observando o uso da memória e o tempo de execução
	\item Disponibilizar bases para uma nova ferramenta para a representação de topologias de máquina
\end{itemize}



\section{Metodologia}
\label{sec:metodologia}

\begin{itemize}
	\item Estudar organização de computadores com foco na hierarquia de memória
	\item Estudar como topologias de máquina são representadas em trabalhos e ferramentas do estado da arte
	\item Entender o protótipo utilizado no Laboratório de Computação Embarcada - \textit{Embedded Computing Lab} (ECL) até o momento
	\item Implementar novos métodos de armazenamento e acesso às informações
	\item Testar os novos métodos e estruturas de dados utilizando máquinas com topologias diferentes e avaliar os resultados
\end{itemize}



\section{Organização do Trabalho}
\label{sec:organizacao_do_trabalho}

As seções restantes estão organizadas da seguinte forma:
No Capítulo \ref{cap:fundamentacao_teorica}, serão apresentados alguns conceitos fundamentais relevantes para o trabalho.
No Capítulo \ref{cap:estado_da_arte}, será fornecida uma visão geral do estado da arte em representação de topologias de máquina.
No Capítulo~\ref{cap:implementacao}, as estruturas e algoritmos desnvolvidos e implementados serão apresentados e analisados e, no Capítulo~\ref{cap:testes}, os resultados dos testes de desempenho realizados com eles.
Por fim, as conclusões e trabalhos futuros serão apresentados no Capítulo \ref{cap:conclusao}.
%No capítulo \ref{cap:andamento_do_trabalho} constam o progresso atual do trabalho e o planejamento do seu prosseguimento.
