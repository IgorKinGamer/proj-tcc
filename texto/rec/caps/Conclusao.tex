\chapter{Conclusão}
\label{cap:conclusao}


O objetivo deste trabalho foi desenvolver formas otimizadas de representar topologias de máquina.
O projeto hwloc, estado da arte em representação de topologias, foi analisado e a função que encontra o ACMP foi identificada como um ponto relevante a se otimizar.
Com essa análise,
novas estruturas foram desenvolvidas, além de um algoritmo utilizando-as, \Novo.
Esse algoritmo e outras abordagens foram implementados e testados.

O \Novo\ mostrou características promissoras, como menor variação no tempo de execução e estruturas que devem originar acessos à memória menos esparsos do que nas outras abordagens.
O \Matriz, apesar de ter tido tempos menores, mesmo sendo um algoritmo de complexidade constante, se mostrou sensível ao aumento do tamanho das árvores utilizadas.
Não foram feitos testes com aplicações reais, mas a variação do tempo observada nos testes realizados é uma evidência de que seu uso da memória prejudicaria o desempenho.
O desempenho do \Simples\ foi semelhante ao do \Novo, mas pode também estar sujeito à influência do padrão de acessos à memória, assim como o \Hwloc, utilizado pelo hwloc, que ainda necessita de um processamento maior.
%O seu uso em uma aplicação real deixaria clara a sua deficiência

Este algoritmo desenvolvido, \Novo, poderia ser utilizado no hwloc, realizando-se as devidas adaptações.
Com base nesta otimização em conjunto com outras que possam ser realizadas, à medida que as estruturas e métodos usados se tornassem incompatíveis com a implementação do hwloc, uma nova ferramenta à parte poderia ser desenvolvida, tendo em foco considerações semelhantes às feitas neste trabalho em relação ao uso da memória.

%SEÇÃO Futuro

Para isso, como trabalhos futuros,
pode-se avaliar o desempenho dos algoritmos diante de aplicações reais
e fazer outras análises, como acompanhar o uso da memória com mais detalhes.
Se isto se mostrar relevante,
outras técnicas podem ser estudadas para diminuir o tamanho das estruturas usadas no \Novo,
incluindo a aplicação de máscaras \texttt{Ou} e \texttt{Ou-Exclusivo} aos IDs na função \Espalha,
para o que há suporte no código desenvolvido, embora não tenha sido usado.

Quanto a outras funções presentes no hwloc, pode-se estudar a possibilidade e o impacto de otimizações na representação do conjunto de CPUs dos objetos, o qual é usado, por exemplo, para determinar se um objeto está sobre outro na hierarquia de memória.
Além disso, outra função identificada como relevante no projeto HieSchella é a \mbox{\texttt{hwloc\_get\_ancestor\_obj\_by\_type}}, que encontra o ancestral em determinado nível, também deixando espaço para se estudar se otimizações teriam efeito significativo no desempenho.

%Melhorias e alternativas não testadas:
%- Eliminar multiplicações (mas testes feitos mostram que não têm grande custo)

%HieSchella - hwloc\_get\_ancestor\_obj\_by\_type
