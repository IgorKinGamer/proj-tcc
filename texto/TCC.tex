\documentclass[a5paper]{ufsc-thesis}

% UTF-8
\usepackage[utf8]{inputenc}
% Fonte
\usepackage[T1]{fontenc}
\usepackage{lmodern}
% Português
\usepackage[brazil]{babel}

%
\usepackage{amsmath}
%
\usepackage{tikz}
\usetikzlibrary{positioning}
%
\usepackage{xparse}

%
\DeclareMathOperator \pai {pai}
\newcommand \id[1]    {\ensuremath{\mathrm{id}_{#1}}}
\newcommand \idstr[1] {\ensuremath{\id{#1}^{\mathrm{str}}}}


%\usepackage{cmap}
%\usepackage[alf]{abntex2cite}

\usepackage[printonlyused]{acronym} 

% Preâmbulo
\titulo{ROTM}
\autor{Eu}
\data{\today}
\instituicao{Universidade Federal de Santa Catarina}
\local{Florianópolis, SC}
\tipotrabalho{TCC}
\orientador{Orientador}
%\coorientador{José {}da Silva Sauro}
\programa{Programa de Graduação em Ciências da Computação}
\preambulo{Trabalho}
\centro{Centro Tecnológico -- CTC}
\assuntos{Ciências da Computação,Modelos,Teses,OpenSource,LaTeX}

\begin{document} %%%

% Inicia parte pré-textual do documento capa, folha de rosto, folha de
% aprovação, aprovação, resumo, lista de tabelas, lista de figuras, etc.
\imprimircapa
\imprimirfolhaderosto*

\begin{siglas}
	\item[tst] tst
	\item[etc] ...
\end{siglas}

%\include{init/pretextuais}
%\clearpage
%\imprimirfichacatalografica
%\tableofcontents

\textual

%\include{rec/caps/introducao}
\chapter{Implementação}

Esta seção apresenta como o projeto hwloc foi analisado, as considerações feitas com relação ao desempenho e as implementações resultantes.
Todo o código foi desenvolvido na linguagem C++.

O hwloc possui diversas funções de percorrimento.
Estas permitem acessar nós da árvore que representa a hierarquia de forma absoluta ou relativa a outros nós.
Por exemplo, é possível encontrar o nó com um determinado índice dentro de um dado nível,
ou, a partir de algum nó, o próximo no mesmo nível.
\tratar{Listar funções}

Essas funções foram analisadas quanto à complexidade com o objetivo de identificar pontos que poderiam ser melhorados do ponto de vista do desempenho.
Essa análise revelou que, em geral, elas têm tempo constante ($\bigO(1)$) ou linear na altura da árvore (\Oalt).
Além disso, foi analisado um projeto de código aberto que utiliza o hwloc, HieSchella \tratar{[ref]},
cujo objetivo é prover portabilidade de performance, característica presente quando se consegue que
uma mesma aplicação rode em diferentes plataformas utilizando os núcleos com eficiência.
Foram identificadas as chamadas mais importantes a funções do hwloc no HieSchella para se ter uma referência
de quais funções são mais relevantes para o desempenho dentro de um projeto real.

Diante dessas considerações, a função que encontra o ancestral comum mais próximo (\fACMP) entre dois nós foi escolhida como alvo de otimizações.
Ela é uma das funções implementadas no hwloc com complexidade \Oalt e está entre as de uso mais significativo no HieSchella.
Na seção a seguir, será discutido como essa função poderia ser implementada de forma mais eficiente e quais as implicações de diferentes abordagens.


\section{Implementações da função \fACMP}

A função que encontra o ancestral comum mais próximo recebe dois nós como entrada (e possivelmente algumas estruturas adicionais, se houver necessidade)
e retorna um nó (o ancestral) como saída.
Cada par de nós em uma árvore tem exatamente um ancestral comum mais próximo.

\subsection{Abordagem inicial}

A maneira provavelmente mais intuitiva de se descobrir o ACMP é ``subir'' pela árvore, isto é, a partir dos dois nós dos quais se deseja encontrar o ACMP,
seguir as ligações em direção aos pais até se chegar ao mesmo nó.
Por ser um método que utiliza apenas a estrutura de árvore em si, sem adição de outras informações, será referido como método simples.
Sua complexidade é \Oalt.
Para que esse método funcione, é necessário subir de forma sincronizada -- a cada passo, devem-se comparar ancestrais dos dois nós iniciais que estejam no mesmo nível.
No caso do hwloc, o algoritmo se torna um pouco mais complicado, pois ele trata hierarquias assimétricas (pode haver ramos sem nó em algum nível).
Neste caso, mesmo que se tenham dois nós no mesmo nível, seus pais podem estar em níveis diferentes.
Um ponto que pode afetar o desempenho deste algoritmo é o fato de que é necessário acessar cada nó
(os nós iniciais e todos os seus ancestrais até o ACMP), e os nós estão espalhados pela memória.

\tratar{(Algoritmo ACMP do hwloc)}

Considerando a estrutura de árvore apenas, a única informação que relaciona um nó aos seus ancestrais
são as ligações entre um nó e seus filhos (ou, no outro sentido, entre cada nó e seu pai).
Assim, todas as sequências de uma ou mais ligações de filho para pai a partir de um nó (ou seja,
entre o nó e seu pai, entre este e o pai dele, e assim por diante) definem os ancestrais do nó.
Isso indica que outras estruturas associadas aos nós ou à árvore como um todo se fazem necessárias
para ser possível encontrar o ACMP com algum método além do simples.
Podemos considerar a seguinte ideia para encontrar outra maneira de implementar a função de ancestral comum:
Para uma dada árvore que representa uma topologia,
\begin{enumerate}
	\item Atribuir um valor (chamado de ID) a cada nó da árvore;
	\item Definir uma função \ACMPIDs que receba o ID de dois nós distintos e tenha como resultado o nó ACMP.
\end{enumerate}
Esses IDs (em conjunto com outras informações associadas a cada nó individualmente ou à árvore como um todo conforme necessário)
podem estabelecer alguma relação entre um nó e seus ancestrais além da que já existe por meio das ligações da árvore.

Idealmente, essa função \ACMPIDs deveria ser \textit{processada} em tempo constante.
Ou seja, é necessário encontrar um algoritmo que seja executado em tempo constante nos processadores modernos.
No entanto, é preciso lembrar que, mesmo que a quantidade de instruções executadas pelo processador seja constante,
a maneira como a memória é acessada pode aumentar o tempo de execução, especialmente quando há outras tarefas fazendo uso da memória, o que deve acontecer em cenários reais.
\tratar{(? Isso pode ser visto... (exemplo à frente com matriz?))}

Com isso em mente, podemos definir tal função usando as operações básicas encontradas no conjunto de instruções das arquiteturas atuais,
tais como as operações aritméticas e operações lógicas bit-a-bit.

\subsection{Matriz}

Uma possibilidade é relacionar cada par de nós ao seu ACMP por meio de uma matriz
em que cada linha representa um nó da árvore, assim como cada coluna, e o cruzamento contém o ACMP entre o nó da linha e o nó da coluna.
Para isso, pode-se atribuir a cada um dos $n$ nós da árvore um ID único entre $0$ e $n-1$ e usar esses IDs como índices na matriz,
que terá, na posição $A(i,j)$, o ACMP entre o nó de ID $i$ e o nó de ID $j$.
No entanto, esta é uma estratégia ingênua, pois esse espaço $\bigO(n^2)$ ocupado na memória resultaria em problemas como sujar a cache da aplicação.
Visto que a matriz é simétrica, apenas cerca de metade dela precisa realmente ser armazenada.
Esta otimização foi utilizada na implementação dos testes de desempenho, conforme o \tratar{[algoritmo X]}.
Isso, no entanto, não altera a complexidade espacial.

% Matriz
\begin{algorithm}[h]
	\caption{\textsc{MatrizACMP} -- \fACMP\ usando uma matriz}
	\label{alg:matriz}
	
	\newcommand \linha {\mathrm{linha}}
	\newcommand \coluna {\mathrm{coluna}}
	
	\Entrada{Dois nós $a$ e $b$ \newline
	         Uma matriz $M$ que possui na linha $i$ e coluna $j$ ($M(i, j)$),
	          $i > j$, o ACMP entre os nós de ID $i$ e $j$}
	\Saida{O ACMP entre $a$ e $b$}
	
	%\BlankLine
	%\tcp{A matriz é simétrica, elementos acima da diagonal principal não são utilizados}
	
	\BlankLine
	%Matriz$\langle$Nó$\rangle$ $M$, Nó \no{a}, Nó \no{b}
	$\linha  \gets \max(\id{a}, \id{b})$ \\
	$\coluna \gets \min(\id{a}, \id{b})$ \\
	\Retorna $M(\linha, \coluna)$ \\
	%\EndFunction
	
\end{algorithm}


\subsection{Função de espalhamento}

Outra possibilidade foi idealizada, dividindo a função \ACMPIDs em dois passos:
\begin{enumerate}
	\item dados os IDs de dois nós, descobrir o ID do ancestral;
	\item encontrar o nó que possui esse ID.
\end{enumerate}
Em linhas gerais, o funcionamento do primeiro passo se baseia no seguinte:
%- O comprimento de um ID é a quantidade de bits da sua representação em binário, descartando zeros à esquerda.
O ID de um nó é formado por um ou mais bits seguidos do ID do seu pai, de modo que, dados dois nós \no{a} e \no{b}, \no{b} descendente de \no{a},
os bits menos significativos de \no{b} são iguais ao ID de \no{a}.
Desse modo, dados dois descendentes de um nó \no{c}, todos os bits menos significativos deles que coincidem
(todos os que vêm antes do primeiro que difere) são iguais ao ID de \no{c} (ignorando zeros à esquerda).
Usando apenas as operações que podem ser vistas \tratar{nos algoritmos [X] e [Y] mais adiante},
para as quais existem instruções que tomam poucos ciclos nas arquiteturas atuais, pode-se descobrir o ID do ACMP.
A quantidade de instruções é fixa, portanto, a complexidade é constante.

Para descrever como os IDs são formados, as seguintes definições são necessárias:
\begin{itemize}
	\item \id{a} é o ID do nó $a$.
	\item \idstr{a} é uma cadeia ({\it string}) de bits correspondente ao \id{a} em binário (com o bit menos significativo na última posição). O tamanho depende do nível, como será especificado adiante.
\end{itemize}
(Para isso,) Os IDs são definidos da seguinte forma:
\begin{itemize}
	\item A raiz tem id 0, e \idstr{\mathit{raiz}} é a cadeia vazia.
	\item Quanto aos demais, para cada nó $a$,
		$$\idstr{a} = x \parallel \idstr{pai(a)}$$
	onde $\parallel$ é a concatenação e $x$ é uma cadeia cujo tamanho é o grau do nível de $\pai(a)$.
	Se a é o i-ésimo filho de seu pai, $x$ possui 1 na i-ésima posição da direta para a esquerda e 0 nas demais.
\end{itemize}

Assim, os IDs de todos os nós de um nível têm o mesmo tamanho, que é o somatório dos graus dos níveis anteriores.
Por exemplo, a [imagem ...] representa os IDs atribuídos a uma árvore:

\begin{figure} \caption{oi} \label{img:exemplo_ids}
\begin{tikzpicture}[no/.style={draw, circle, minimum size=1cm, align=center}]
%[estado/.style={circle,minimum size=1cm}]
%\node [estado] (IUA)    at ({360/\n * (3+.5)}:\raio) {I U A};
	% Árvore
	\node [no] (a) {$a$}
		[level distance=3cm,
		level 1/.style={sibling distance=45mm},
		level 2/.style={sibling distance=22mm}]
		child {node [no] (b) {b}
			child {node [no] (e) {e}}
			child {node [no] (f) {f}} }
		child {node [no] (c) {c}
			child {node [no] (g) {g}}
			child {node [no] (h) {h}} }
		child {node [no] (d) {d}
			child {node [no] (i) {i}}
			child {node [no] (j) {j}} }
	;
	% IDs
	\newcommand \caa[1] {\textcolor{red!50!black}{#1}}
	\newcommand \cab[1] {\textcolor{red!50!red}{#1}}
	\newcommand \cac[1] {\textcolor{red!50!white}{#1}}
	\newcommand \cba[1] {\textcolor{blue!75!black}{#1}}
	\newcommand \cbb[1] {\textcolor{blue!75!white}{#1}}
	\newcommand \labelId[4][right]
		{\node [align=center, node distance=0, #1=of #2]
		{$\id{#2} = #3$\\$\idstr{#2} = #4$}}
	\newcommand \labelIdF[3] {\labelId[below]{#1}{#2}{#3}}
	\labelId{a}{0}{{\color{gray} (vazio)}};
		\labelId{b}{1}{\caa{001}};
			\labelIdF{e}{9}{\cba{01}\ \caa{001}};
			\labelIdF{f}{17}{\cbb{10}\ \caa{001}};
		\labelId{c}{2}{\cab{010}};
			\labelIdF{g}{10}{\cba{01}\ \cab{010}};
			\labelIdF{h}{18}{\cbb{10}\ \cab{010}};
		\labelId{d}{4}{\cac{100}};
			\labelIdF{i}{12}{\cba{01}\ \cac{100}};
			\labelIdF{j}{20}{\cbb{10}\ \cac{100}};
\end{tikzpicture}
\end{figure}
 % Explicação de como os IDs são formados

Falta, então, apenas o segundo passo, o de encontrar o nó a partir do ID.
Algumas opções para isso seriam:
\begin{itemize}
	\item Usar os IDs como índices em um arranjo: Seria simples, mas impraticável -- poderiam ser necessários arranjos com milhões de posições (devido a como os IDs são formados) e apenas algumas centenas ocupadas.
	\item Usar uma função de espalhamento (\textit{hash}): a maneira mais simples seria apenas aplicar a operação módulo com algum $m$ ($\mathrm{id} \bmod m$).
	No entanto, podem haver colisões (dois IDs diferentes podem ser congruentes módulo $m$).
	Isso pode ser tratado, mas acarretaria acessos adicionais à memória, o que não é desejável.
	\tratar{Exemplo}
	\item Buscar uma função que não cause colisões: o mesmo que a função de espalhamento, porém utilizando um valor de módulo que não cause colisões.
	Pode exigir arranjos cujo tamanho é algumas vezes maior que a quantidade de nós, mas aparenta compensar quando comparado a tratar colisões de outros modos.
	\tratar{Exemplo}
	Este foi o método escolhido.
\end{itemize}

O \tratar{Algoritmo X} apresenta como esta 

\tratar{[Algoritmo]}

\tratar{Estruturas foram testadas}

%Detalhes de como os IDs são formados
\tratar{Pode falhar se os dois nós de entrada são, na verdade, o mesmo nó. If resolve
Funciona em casos de hierarquias assimétricas, simples não
Limitação: Quantidade de bits necessária - Solução parcial: 'Esconder' níveis cujos nós têm só um filho (níveis de grau um).}



%SEÇÃO Futuro

%Melhorias e alternativas não testadas:
%- Eliminar multiplicações (mas testes feitos mostram que não têm grande custo)
%- Todos os nós em um só arranjo
%- Máscaras Ou e OuExclusivo; Outras maneiras de diminuir o tamanho

%HieSchella - hwloc\_get\_ancestor\_obj\_by\_type

%% Seção sobre testes de desempenho

\chapter{Testes de desempenho}
%POR QUÊ

Com o objetivo de avaliar o desempenho do algoritmo desenvolvido, foram realizados testes
que permitiram comparar o desempenho das diferentes abordagens.
Foi medido o tempo tomado pelos algoritmos
ao se encontrar repetidamente o ACMP entre os nós folhas de uma árvore.
Com esses dados, analisados

Um programa foi desenvolvido, também na linguagem C++, para realizar os testes de desempenho.
Foi utilizada a funcionalidade de templates da linguagem para facilitar a definição equivalente dos testes para todos os algoritmos, mas ainda assim permitir que o compilador otimizasse as chamadas, evitando custos adicionais durante os testes devido à hierarquia de classes utilizada.
O programa depende da biblioteca do hwloc.
Os valores medidos são escritos em um arquivo no formato CSV (Valores Separados por Vírgula -- \textit{Comma-Separated Values})


\section{Máquinas utilizadas nos testes}

Os testes foram executados sobre duas máquinas (notebooks), que serão identificadas como Máquina A e Máquina B, para as quais o programa lstopo gera as representações de hierarquia apresentadas na Figura \ref{img:maquinas}. Estas representações revelam o tamanho das memórias cache, que podem ter efeito nos resultados dos testes. Ambas possuem o sistema operacional Windows 10. \tratar{Mais detalhes (processador)?}

\begin{figure}
	\centering
	\caption{Saída do programa lstopo sobre as máquinas utilizadas}
	\label{img:maquinas}
	\begin{subfigure}{.4\textwidth}
		\caption{Máquina A}
		\includegraphics[width=\textwidth]{rec/img/MaqA}
	\end{subfigure}
	~
	\begin{subfigure}{.4\textwidth}
		\caption{Máquina B}
		\includegraphics[width=\textwidth]{rec/img/MaqB}
	\end{subfigure}

\end{figure}

%COMO
\section{Estrutura dos testes}

Os testes consistem em encontrar repetidamente o ACMP entre os nós folhas de uma dada árvore com os diferentes algoritmos -- o existente no hwloc e os implementados.
Cada vez que as estruturas necessárias são criadas e os testes são executados sobre elas com um determinado algoritmo, obtém-se uma observação, que é o tempo que levou para realizar a quantidade especificada de repetições da função ACMP entre os nós folhas da árvore com o algoritmo.

Para cada algoritmo, uma árvore de estrutura equivalente é criada (as estruturas de dados dependem do algoritmo) a partir dos graus fornecidos como entrada.
%(Como os graus são usados para montar a árvore -- se não for explicado antes)
Os graus são recebidos como uma lista de inteiros positivos.
O i-ésimo será o grau do nível $i-1$.
Deste modo, se os graus recebidos são $(g_1, g_2, ..., g_n)$, a raiz terá $g_1$ filhos, cada um com $g_2$ filhos, e assim por diante, até os nós do nível $n-1$, que terão $g_n$ filhos cada.

Para obter uma observação de um algoritmo, uma lista contendo todos os pares possíveis de nós folhas (todas as combinações de duas folhas) é criada.
Esta lista é embaralhada, usando uma semente fixa, de modo que a ordem pseudo-aleatória é a mesma para todas as execuções de todos os algoritmos sobre esta árvore.

Uma observação é obtida por meio de uma etapa de aquecimento seguida de uma de medição, na qual o tempo total das repetições da função ACMP é medido.
Ambas as etapas consistem em algum número de rodadas, o qual geralmente deve estar na casa de alguns milhares, dependendo do tamanho da árvore, para que o tempo medido seja significativo.
Em cada rodada, a lista previamente embaralhada de pares de folhas é varrida e, para cada par, o ACMP é encontrado usando o algoritmo em questão.

As observações dos diferentes algoritmos são realizadas de forma intercalada.
O número de observações obtidas para cada algoritmo em uma execução do programa depende de dois parâmetros, número de iterações externas e de iterações internas.
O número de iterações internas é a quantidade de observações que serão obtidas para um dos algoritmos antes de passar para outro.
A execução da quantidade de iterações internas para cada algorimo compõe uma iteração externa.
Esta forma de especificar a quantidade de observações originou-se nas etapas iniciais dos testes, para facilitar a visualização dos resultados, mas foi mantida.
No entanto, julga-se melhor usar poucas iterações internas e mais externas para evitar que eventuais condições temporárias da máquina, causadas por elementos externos ao programa, afetem diversas observações de apenas um dos algoritmos.

O programa também permite escolher quais algoritmos serão testados.
Os possíveis são o simples, o que utiliza as novas estruturas, o que utiliza uma matriz e o implementado no hwloc.

%(Tabela com os parâmetros?)


\subsection{RESULTADOS}

As árvores utilizadas nos testes correspondem à hierarquia de memória de máquinas reais, apresentadas no site do projeto hwloc, como representadas pelo programa lstopo \cite{lstopo}.
Assim, os resultados refletem a diferença dos algoritmos quando operando sobre hierarquias reais.
%\tratar{(Referência: The Best of lstopo - https://www.open-mpi.org/projects/hwloc/lstopo/)}

\tratar{especificar parâmetros utilizados}

Inicialmente, os testes foram feitos sem acessar o ACMP obtido a cada execução, ou seja, para cada par de folhas, simplesmente se encontrava o ponteiro para o ancestral, mas nenhum dado do ancestral era obtido.
%(apenas era feito algo com o ponteiro retornado para evitar otimizações que descartassem tudo devido ao resultado não ser usado).
No entanto, em cenários reais, os nós seriam buscados para se obter alguma informação sobre eles, portanto, foi adicionado um acesso a um valor qualquer de cada ACMP encontrado para simular isso.
A Figura \ref{fig:box_sem_com} compara o desempenho dos quatro algoritmos na Máquina A para a árvore com os graus $(1, 4, 1, 1, 9, 2, 1, 1, 4)$ antes e depois do acréscimo desse acesso.
Sem o acesso foram feitas 300 observações de cada algoritmo e, com o acesso, 150 observações.
Como era esperado, os tempos aumentaram para todos os algoritmos, porém o algoritmo menos afetado foi o novo e o mais afetado foi o da matriz, considerando a porcentagem de aumento do valor mediano após acrescentar o acesso, conforme a Tabela \ref{tab:aumento_mediana}.

\begin{figure}
	\caption{Tempos com e sem acesso ao ACMP}
	\label{fig:box_sem_com}
	\resizebox{\textwidth}{!}{\input{rec/img/opa}}
\end{figure}

\begin{table}[htb]
	\IBGEtab{\caption{Aumento (\%) do tempo mediano}%
		\label{tab:aumento_mediana}}
	{\begin{tabular}{cccc}
		\toprule
		Simples & Novo & hwloc & Matriz \\
		\midrule 
		14,04 & 10,08 & 14,92 & 27,93 \\
		\bottomrule
	\end{tabular}}{}%  \fonte{Produzido pelos autores.}%
\end{table}

%Comparação - O desempenho do algoritmo desenvolvido foi comparado com o do algoritmo simples, o do hwloc (e o que usa a matriz)(?)

%Outras medições?

\tratar{Outras análises: comparações entre as máquinas, comparar tempo médio por acesso do novo algoritmo entre topologias de tamanhos diferentes}


\bibliography{ref}

\end{document}