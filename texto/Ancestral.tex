\chapter{Implementação}

(Conceitos:
- Grau de um nível: o maior entre os graus dos nós de um nível.
É comum todos os graus dos nós de um nível serem iguais, mas, em hierarquias assimétricas, isso não ocorre.)

O hwloc possui diversas funções de percorrimento.
Estas permitem acessar nós da árvore que representa a hierarquia de forma absoluta ou relativa a outros nós.
Por exemplo, é possível encontrar o nó com um determinado índice dentro de um dado nível,
ou, a partir de algum nó, o próximo no mesmo nível.
*

Essas funções foram analisadas quanto à complexidade com o objetivo de identificar pontos que poderiam ser melhorados do ponto de vista do desempenho.
Essa análise revelou que, em geral, elas têm tempo constante ($O(1)$) ou linear na altura da árvore ($O(altura)$).
Além disso, foi analisado um projeto de código aberto que utiliza o hwloc, HieSchella [ref],
cujo objetivo é prover portabilidade de performance, característica presente quando se consegue que
uma mesma aplicação rode em diferentes plataformas utilizando os núcleos com eficiência.
Foram identificadas as chamadas mais importantes a funções do hwloc no HieSchella para se ter uma referência
de quais funções são mais relevantes para o desempenho dentro de um projeto real.

Diante dessas considerações, a função que encontra o ancestral comum mais próximo entre dois nós foi escolhida como alvo de otimizações.
Ela é uma das funções implementadas no hwloc com complexidade $O(altura)$ e está entre as de uso mais significativo no HieSchella.
Na seção a seguir, será discutido como essa função poderia ser implementada de forma mais eficiente e quais as implicações de diferentes abordagens.


\section{Implementações da função ACMP}

A função que encontra o ancestral comum mais próximo recebe dois nós como entrada (e possivelmente algumas estruturas adicionais, se houver necessidade)
e retorna um nó (o ancestral) como saída.
Cada par de nós em uma árvore tem exatamente um ancestral comum mais próximo.

(Definir ancestral e descendente?)

\subsection{Abordagem inicial}

A maneira provavelmente mais intuitiva de se descobrir o ACMP é "subir" pela árvore, isto é, a partir dos dois nós dos quais se deseja encontrar o ACMP,
seguir as ligações em direção aos pais até se chegar ao mesmo nó.
Por ser um método que utiliza apenas a estrutura de árvore em si, sem adição de outras informações, será referido como método simples.
Sua complexidade é $O(altura)$.
Para que esse método funcione, é necessário subir de forma sincronizada -- a cada passo, devem-se comparar ancestrais dos dois nós iniciais que estejam no mesmo nível.
No caso do hwloc, o algoritmo se torna um pouco mais complicado, pois ele trata hierarquias assimétricas (pode haver ramos sem nó em algum nível).
Neste caso, mesmo que se tenham dois nós no mesmo nível, seus pais podem estar em níveis diferentes.
Um ponto que pode afetar o desempenho deste algoritmo é o fato de que é necessário acessar cada nó
(os nós iniciais e todos os seus ancestrais até o ACMP), e os nós estão espalhados pela memória.

(Algoritmo ACMP do hwloc)
(Algoritmo ACMP do hwloc)

Considerando a estrutura de árvore apenas, a única informação que relaciona um nó aos seus ancestrais
são as ligações entre um nó e seus filhos (ou, no outro sentido, entre cada nó e seu pai).
Assim, todas as sequências de uma ou mais ligações de filho para pai a partir de um nó (ou seja,
entre o nó e seu pai, entre este e o pai dele, e assim por diante) definem os ancestrais do nó.
Isso indica que outras estruturas associadas aos nós ou à árvore como um todo se fazem necessárias
para ser possível encontrar o ACMP com algum método além do simples.
Podemos considerar a seguinte ideia para encontrar outra maneira de implementar a função de ancestral comum:
Para uma dada árvore que representa uma topologia:
- Atribuir um valor (chamado de ID) a cada nó da árvore
- Definir uma função $ACMP_{IDs}$ que receba o ID de dois nós distintos e tenha como resultado o nó ACMP.
Esses IDs (em conjunto com outras informações associadas a cada nó individualmente ou à árvore como um todo conforme necessário)
podem estabelecer alguma relação entre um nó e seus ancestrais além da que já existe por meio das ligações da árvore.

Idealmente, essa função $ACMP_{IDs}$ deveria ser *processada* em tempo constante.
Ou seja, é necessário encontrar um algoritmo que seja executado em tempo constante nos processadores modernos.
No entanto, é preciso lembrar que, mesmo que a quantidade de instruções executadas pelo processador seja constante,
a maneira como a memória é acessada pode aumentar o tempo de execução, especialmente quando há outras tarefas fazendo uso da memória, o que deve acontecer em cenários reais.
(? Isso pode ser visto... (exemplo à frente com matriz?))

Com isso em mente, podemos definir tal função usando as operações básicas encontradas no conjunto de instruções das arquiteturas atuais,
tais como as operações aritméticas e operações lógicas bit-a-bit.

\subsection{Matriz}

Uma possibilidade é relacionar cada par de nós ao seu ACMP por meio de uma matriz
em que cada linha representa um nó da árvore, assim como cada coluna, e o cruzamento contém o ACMP entre o nó da linha e o nó da coluna.
Para isso, pode-se atribuir a cada um dos n nós da árvore um ID único entre 0 e n-1 e usar esses IDs como índices na matriz,
que terá, na posição A(i,j), o ACMP entre o nó de ID i e o nó de ID j.
No entanto, esta é uma estratégia ingênua, pois esse espaço $O(n^2)$ ocupado na memória resultaria em problemas como sujar a cache da aplicação.
Visto que a matriz é simétrica, apenas cerca de metade dela precisa realmente ser armazenada.
Esta otimização foi utilizada na implementação dos testes de desempenho, conforme o [algoritmo X].
Isto, no entanto, não altera a complexidade espacial.

(Algoritmo matriz)

\subsection{Função de espalhamento}

Outra possibilidade foi idealizada, dividindo a função $ACMP_{IDs}$ em dois passos:
- dados os IDs de dois nós, descobrir o ID do ancestral,
- encontrar o nó que possui esse ID.
Em linhas gerais, o funcionamento do primeiro passo se baseia no seguinte:
- O comprimento de um ID é a quantidade de bits da sua representação em binário, descartando zeros à esquerda.
- O ID de um nó é formado por um ou mais bits seguidos do ID do seu pai, de modo que, dados dois nós a e b, b descendente de a,
os n bits menos significativos de b são iguais ao ID de a, onde n é o comprimento do ID de a.
Desse modo, dados dois descendentes de um nó c, todos os bits menos significativos deles que coincidem
(todos os que vêm antes do primeiro que difere) são iguais ao ID de c (ignorando zeros à esquerda).
Usando apenas as operações que podem ser vistas nos algoritmos [X] e [Y],
para as quais existem instruções que tomam poucos ciclos nas arquiteturas atuais, pode-se descobrir o ID do ACMP
- A quantidade de instruções é fixa, portanto, a complexidade é constante

Para descrever como os IDs são formados, as seguintes definições são necessárias:
\begin{itemize}
	\item \id{a} é o ID do nó $a$.
	\item \idstr{a} é uma cadeia ({\it string}) de bits correspondente ao \id{a} em binário (com o bit menos significativo na última posição). O tamanho depende do nível, como será especificado adiante.
\end{itemize}
(Para isso,) Os IDs são definidos da seguinte forma:
\begin{itemize}
	\item A raiz tem id 0, e \idstr{\mathit{raiz}} é a cadeia vazia.
	\item Quanto aos demais, para cada nó $a$,
		$$\idstr{a} = x \parallel \idstr{pai(a)}$$
	onde $\parallel$ é a concatenação e $x$ é uma cadeia cujo tamanho é o grau do nível de $\pai(a)$.
	Se a é o i-ésimo filho de seu pai, $x$ possui 1 na i-ésima posição da direta para a esquerda e 0 nas demais.
\end{itemize}

Assim, os IDs de todos os nós de um nível têm o mesmo tamanho, que é o somatório dos graus dos níveis anteriores.
Por exemplo, a [imagem ...] representa os IDs atribuídos a uma árvore:

\begin{figure} \caption{oi} \label{img:exemplo_ids}
\begin{tikzpicture}[no/.style={draw, circle, minimum size=1cm, align=center}]
%[estado/.style={circle,minimum size=1cm}]
%\node [estado] (IUA)    at ({360/\n * (3+.5)}:\raio) {I U A};
	% Árvore
	\node [no] (a) {$a$}
		[level distance=3cm,
		level 1/.style={sibling distance=45mm},
		level 2/.style={sibling distance=22mm}]
		child {node [no] (b) {b}
			child {node [no] (e) {e}}
			child {node [no] (f) {f}} }
		child {node [no] (c) {c}
			child {node [no] (g) {g}}
			child {node [no] (h) {h}} }
		child {node [no] (d) {d}
			child {node [no] (i) {i}}
			child {node [no] (j) {j}} }
	;
	% IDs
	\newcommand \caa[1] {\textcolor{red!50!black}{#1}}
	\newcommand \cab[1] {\textcolor{red!50!red}{#1}}
	\newcommand \cac[1] {\textcolor{red!50!white}{#1}}
	\newcommand \cba[1] {\textcolor{blue!75!black}{#1}}
	\newcommand \cbb[1] {\textcolor{blue!75!white}{#1}}
	\newcommand \labelId[4][right]
		{\node [align=center, node distance=0, #1=of #2]
		{$\id{#2} = #3$\\$\idstr{#2} = #4$}}
	\newcommand \labelIdF[3] {\labelId[below]{#1}{#2}{#3}}
	\labelId{a}{0}{{\color{gray} (vazio)}};
		\labelId{b}{1}{\caa{001}};
			\labelIdF{e}{9}{\cba{01}\ \caa{001}};
			\labelIdF{f}{17}{\cbb{10}\ \caa{001}};
		\labelId{c}{2}{\cab{010}};
			\labelIdF{g}{10}{\cba{01}\ \cab{010}};
			\labelIdF{h}{18}{\cbb{10}\ \cab{010}};
		\labelId{d}{4}{\cac{100}};
			\labelIdF{i}{12}{\cba{01}\ \cac{100}};
			\labelIdF{j}{20}{\cbb{10}\ \cac{100}};
\end{tikzpicture}
\end{figure}


Falta, então, apenas o segundo passo, o de encontrar o nó a partir do ID.
- Usar os IDs como índices em um arranjo seria simples, mas impraticável:
poderiam ser necessários arranjos com milhões de posições (devido a como os IDs são formados) e apenas algumas centenas ocupadas.
- Usar uma função de espalhamento (hash): a maneira mais simples seria apenas aplicar a operação módulo com algum m (id mod m).
No entanto, podem haver colisões (dois IDs diferentes podem ser congruentes módulo m).
Isso pode ser tratado, mas acarretaria acessos adicionais à memória, o que não é desejável.
- Buscar uma função que não cause colisões: o mesmo que a função de espalhamento, porém utilizando um valor de módulo que não cause colisões.
Pode exigir arranjos cujo tamanho é algumas vezes maior que a quantidade de nós, mas aparenta compensar quando comparado a tratar colisões de outros modos.
Este foi o método escolhido.



Detalhes de como os IDs são formados
Função (algoritmo) em si
Pode falhar se os dois nós de entrada são, na verdade, o mesmo nó. If resolve
Funciona em casos de hierarquias assimétricas, simples não
Limitação: Quantidade de bits necessária - Solução parcial: 'Esconder' níveis cujos nós têm só um filho (níveis de grau um).


Código em C++


SEÇÃO Futuro

Melhorias e alternativas não testadas:
- Eliminar multiplicações (mas testes feitos mostram que não têm grande custo)
- Todos os nós em um só arranjo
- Máscaras Ou e OuExclusivo

HieSchella - hwloc\_get\_ancestor\_obj\_by\_type
